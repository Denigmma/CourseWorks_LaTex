% !TeX spellcheck = ru_RU
% !TEX root = vkr.tex

\section{Постановка задачи}
\label{sec:task}

Целью работы является расширение функциональности пакета DEGANN\cite{degann} --- путём внедрения рекуррентных нейронных сетей с архитектурой GRU (Gated Recurrent Unit)\cite{gru}.

Для её выполнения были поставлены следующие задачи:
\subsubsection{Общая архитектура решения}
\begin{enumerate}
    \item Реализация модуля с топологией рекуррентной сети со слоями tensorflow для архитектуры GRU и внедрение его в пакет DEGANN (раздел~\ref{subsec:topology});
    \item Создание модели с архитектурой GRU. Создание и интеграция callback-функции для расширения функциональности обучения и его трекинга (раздел~\ref{subsec:model_callbacks});
\end{enumerate}

\subsubsection*{Валидация решения}
\begin{enumerate}
    \item Создание датасета для корректной передачи данных в модель. Реализация сложно-аппроксимируемой функции для тестирования (раздел~\ref{subsec:dataset});
    \item Подбор метрики, настройка гиперпараметров (раздел~\ref{subsec:metrics});
\end{enumerate}

А так же обучить модель, продемонстрировать результаты в качестве сравнения значений лосс-функций и графиков аппроксимации.