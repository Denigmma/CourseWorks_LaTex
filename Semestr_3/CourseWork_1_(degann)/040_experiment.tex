% !TeX spellcheck = ru_RU
% !TEX root = vkr.tex

\section{Эксперимент}
Цель экмперемента - обучить и сравнить модели с полносвязной и рекуррентной архитектурами с одинаковым количеством слоев и нейронов в качестве аппроксимации встроенных в DEGANN функций, а так же на реализованной сложно-аппроксимируемой функции. Сравнить время обучения, показания лосс функций, а так же качество аппроксимации.

\subsection{Условия эксперимента}
\begin{itemize}
    \item Количество слоев: 5
    \item Количество нейронов в каждом слое: 30
    \item Количество эпох (несколько тестов):
    \begin{itemize}
        \item 100
        \item 200
        \item 500
        \item 1000
    \end{itemize}
\end{itemize}
\subsection{Программные характиристики вычислительной машины}
\begin{itemize}
    \item Процессор: Intel i5 1240p
    \item Видеочип: Iris Xe Graphics G7 80EUs
    \item ОЗУ: 16 гб
\end{itemize}

\subsection{Исследовательские вопросы}
\begin{description}
    \item[RQ1]: Лучше ли рекуррентная  сеть справляется с аппроксимацией уравнений, встроенных в DEGANN?
    \item[RQ2]: Насколько существенна разница в аппроксимации рекуррентной  сети и полносвязной?
\end{description}

\subsection{Итоги}
На данном этапе работы, внедренная архитектура находится на этапе тестирования. Эксперимент еще не до конца завершен и его содержание еще не до конца написано.

%\subsection{Метрики}
%%
%%Как мы сравниваем, что результаты двух подходов лучше или хуже:
%%\begin{itemize}
%%    \item Производительность.
%%    \item Строчки кода.
%%    \item Как часто алгоритм \enquote{угадывает} правильную класси\-фикацию входа.
%%\end{itemize}
%%
%%\noindent Иногда метрики вырожденные (да/нет), это не очень хорошо, но если в области исследований так принято, то ладно.
%%Если метрики хитрые (даже IoU или $F_1$-меру можно считать хитрыми), разберите их в обзоре, пояснив, почему выбраны именно такие метрики.
%
%\subsection{Результаты}
%
%%Результаты понятно что такое.
%%Тут всякие таблицы и графики, как в таблице \ref{time_cmp_obj_func}.
%%Обратите внимание, как цифры выровнены по правому краю, названия по центру, а разделители $\times$ и $\pm$ друг под другом.
%%
%%Перед написанием данного раздела имеет смысл проконсультироваться с литературой по проведению экспериментов~\cite{SmirnovCheatsheet}.
%%
%%Скорее всего Ваши измерения будут удовлетворять нормальному распределению, в идеале это надо проверять с помощью критерия Кол\-могорова и т.п.
%%Если критерий этого не подтверждает, то у Вас что-то сильно не так с измерениями, надо проверять кэши процессора, отключать Интернет во время измерений, подкручивать среду исполне\-ния (англ. runtime), что\-бы сборка мусора не вмешивалась и т.п.
%%Если критерий удовлетворён, то необходимо либо указать мат. ожидание и доверительный/предсказы\-вающий интервал, либо мат. ожидание и среднеквадратичное отклонение, либо, если совсем лень заморачиваться, написать, что все измерения проводились с погрешностью, например, в 5\%.
%%Не приводите слишком много значащих цифр (например, время работы в 239.1 секунды при среднеквадратичном отклонении в 50 секунд выглядит глупо, даже если ваш любимый бенчмарк так посчитал).
%%
%%Замечание: если у вас получится улуч\-шение производительности в пределах погреш\-ности, то это обязательно вызовет вопросы.
%%Если погрешность получилась значительной (больше 10-15\% от среднего), это тоже вызовет вопросы, на которые надо ответить, либо разобравшись, что не так (первый подозреваемый~--- мультимодальное распределение), либо более глубоко статистически проанализировав результаты (например, привести гистограммы).
%%
%%В этом разделе надо также явно ответить на Research Questions или как-то их прокомментировать.
%
%\subsubsection{RQ1} Пояснения
%\subsubsection{RQ2} Пояснения
%
%\subsection{Обсуждение результатов}
%
%%Чуть более неформальное обсуждение, то, что сделано.
%%Например, почему метод работает лучше остальных?
%%Или, что делать со случаями, когда метод классифицирует вход некорректно.
%
%
%\subsection{Воспроизводимость эксперимента}
%
%Это настолько важно, что заслуживает тут своего подраздела (в самой работе не надо, это должно естественно вытекать из разделов выше)~--- эксперимент должно быть можно повторить и получить примерно такие же результаты, как у Вас.
%Поэтому выложите свой код, которым мерили.
%Выложите данные, на которых мерили.
%Или напишите, где эти данные взять.
%Напишите, как конкретно запустить, в каком окружении, что надо дополнительно поставить и т.п.
%Чтобы любой второкурсник мог выполнить все пункты и получить тот же график/таблицу, что у Вас.
%Не обязательно это делать прямо в тексте, можете в README своего репозитория, но где-то надо.
